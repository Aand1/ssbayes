%%
%% abstract.tex - Memoria de la tesis

\cleardoublepage
\thispagestyle{empty}
\addsec*{\protect\centering Abstract}
\markboth{Abstract}{}

Dynamic state estimation is one of the problem on mobile robotics that has received a lot of research efforts lately, especially on applications related with localization.
The aim of this project is to study a series of methods with solid foundations on statistics, called Bayesian filters.
Within this type of filters, we will study those known as Kalman filters.
In particular we will consider four types of filters, each of them designed to deal with particular situations.
The algorithms assessed in this project will be: the classic Kalman filter, the extended Kalman filter, the unscented Kalman filter and the cubature Kalman filter.
In addition, we will study a MATLAB implementation to find which one yields better estimates on the localization of a mobile robot.
To determine which is the best filter we will design a series of experiments that will help us to discover the weak points of each one and which gives the best performance.
For the implementation of the experiments we will use V-REP, a program that allows us to simulate real robots and their dynamics, sensors, actuators and their interactions with different environments.
Finally, with the help of this tool, we will obtain important conclusions about which is the best filter and assess if this is related with the theoretical presentation made in the following chapters.
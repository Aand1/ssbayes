%%
%% conclusiones.tex - Memoria de la tesis
%%
%%   Copyright 2009-2010 Jesús Torres <jmtorres@ull.es>
%%
%% Esta obra está bajo licencia Creative Commons Reconocimiento 3.0 Unported
%%
\pagestyle{scrheadings}
\ihead[]{\rightmark}
\ohead[]{Iván Rodríguez Méndez}
\ofoot[]{\thepage{}}
\addchap{Conclusions and future work}
In this project we have developed a prototyping and simulation framework based on V-REP and MATLAB, which has demonstrated being a suitable combination of a dynamic robot simulator with a fast prototyping experimental environment.
By using a toolbox with both well--known and state--of--the--art implementations of Kalman filter algorithms~\cite{toolbox_simo}, as introduced in chapter~\ref{ch:capitulo3}, we were able to focus on designing and developing an experimental framework.
Using such framework we conducted some experiments to assess the impact of different key concepts related with mobile robot localization, and the issues that classic and novel Kalman filter algorithms suffer.

Regarding the experiments, we have observed that the algorithm showing a better performance on average was the \ac{CKF}, which has demonstrated being a state--of--the--art approach as shown in~\cite{zhang_cubature_2013}.
Furthermore, we have found that \ac{UKF} is a very strong method in many situations, but with a big weakness regarding the singularities which appear with poor information from sensors.
The \ac{EKF} has also proven to be a filter with a decent behavior, and while not standing out over the others, its performance tend to be enough for most situations.
Finally, the \ac{KF} was the filter with the worst performance of the four algorithms studied, as expected.
However, after the experiments conducted, we have seen that even if its estimates are worse that the ones of the other three filters, its performance is not too bad on situations with low slippery degree.
On the contrary, as we saw in the experiment on a complex path, if the slippery degree is large, the estimation of the \ac{KF} becomes totally wrong.

Although experiments have brought us some expected results, they have shed light on the actual version of the filter which would yield the best results, depending on the system we want to use with.
We have seen that Gaussian filters and especially Kalman filters are very sensitive to modeling errors.
For this reason, depending on the actual system which we are working with, we should apply a filter over others.
It is clear that if our system is very simple, where an \ac{EKF} would be enough, we should not waste resources trying to implement an \ac{UKF} or an \ac{CKF}.
On the other hand, if we have a real system with high nonlinearities or a large number of states, the most suitable implementation seems to be the \ac{CKF}.

For the \ac{CKF} and the \ac{UKF} we have proposed in chapter~\ref{ch:capitulo5} a modification on the odometry model.
As we have seen in the experiments, the implementation of these modifications has been a success since filters have obtained lower error in the estimation.

Thus, as the final conclusions, firstly, the application of any of these state--of--the--art filters (\ac{EKF}, \ac{UKF} and \ac{CKF}) depends on the complexity of our system, the modelization error of the disturbances, and the computational resources available.
Secondly, the toolbox used in this project has shown to be a great starting point for implementing localization experiments for mobile robots using the Kalman filtering theory.
Finally, the framework we have developed could be helpful for deploying simulated and real experiments to compare the algorithms described with other advanced localization algorithms, which could also be used for teaching purposes.
The resulting software has been published as a public repository on GitHub~\cite{_tfg_repo}.

\addsec{Future work}

It is obvious that using a simulator like V-REP eases the implementation.
However, there are many possible improvements regarding the implementation of the functions, as for example the memory allocation to gain speed in executing some parts of the code.
Furthermore, a faster implementation of the interaction with the simulator can be done in another programming language such as C++, Java, or Python, keeping the portability between different platforms.
Another interesting point to invest additional efforts could be the development of a graphical interface in which simulation parameters can be tuned.
Also, an improved representation of the results can be explored.

We could also consider the possibility of generating the position's objectives automatically, i.e. a random generator of each target.
A simillar implementation for the landmark placement could be done.

From the model point of view, the API supports importing any type of model and the modification of its physical properties, e.g., the size of the wheels or the frame of the robot.

As we can see, there is much work ahead on this project, which can also deploy advanced methods like particle filters and many more algorithms for estimating the position of a mobile robot.
The main idea after studying and implementing this Kalman filtering experimental framework is to allow for a platform which could be extended and become a tool to simulate different solutions to the problem of dynamic state estimation.

